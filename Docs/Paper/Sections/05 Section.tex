
This study provides valuable insights into the impact of classroom gender composition on students' university major decisions. Our analysis reveals significant variations in female students' choices across academic disciplines, influenced by the presence of male peers in secondary school classrooms.

These findings underscore the complex interplay of gender dynamics in educational environments. Specifically, a higher proportion of male students correlates with increased interest among female students in traditionally male-dominated fields like engineering, architecture, and mathematics. Conversely, greater gender diversity in classrooms leads female students to exhibit a stronger preference for humanities and social sciences.

These results suggest that exposure to male peers may foster competitiveness and aspirations among female students, potentially narrowing gender gaps in STEM fields. However, they also highlight enduring gender-stereotypical preferences, with female students leaning towards communication-oriented disciplines in classrooms with more male representation.

Interestingly, minimal gender differences are observed in fields such as agriculture, veterinary sciences, and medicine, indicating limited influence of classroom gender compositions on major selection within these domains.

The findings underscore the importance of early educational experiences in shaping career trajectories and addressing gender disparities. By cultivating inclusive and gender-balanced learning environments, educational institutions can mitigate stereotypical pressures and promote equitable distribution of interests among students.

Moving forward, further research is warranted to elucidate the multifaceted mechanisms underlying these patterns, including peer influences, role modeling, and sociocultural norms. Longitudinal studies tracking students' evolving aspirations could provide deeper insights into the dynamics of gender composition and its long-term effects.

In conclusion, our analysis highlights the nuanced relationship between classroom gender composition and students' post-secondary schooling decisions. While specific majors exhibit varied responses to gender dynamics, overall, our findings suggest that gender-diverse environments may encourage female students to pursue higher education opportunities beyond secondary school.
