% \section{Introduction}\label{section:01}

% The underrepresentation of women in science, technology, engineering, and mathematics (STEM) fields is a persistent global challenge. Studies consistently highlight the substantial income advantage associated with STEM careers, making these fields highly desirable. For instance, \citet{nsf2023} reports a 35\% income advantage for STEM careers in 2023. However, despite these incentives, women remain less likely than men to pursue STEM education and careers. This disparity limits women's economic opportunities and deprives STEM fields of valuable talent. This paper examines whether the gender composition of classrooms influences female students' choices of university majors, particularly in STEM fields. 

% We focus on the case of Colombia, where a significant gender gap persists in STEM fields despite women achieving higher overall university graduation rates. Data from the Instituto Colombiano para la Evaluación de la Educación \citep{ICFES2019} reveals that only 16\% of female university graduates choose STEM careers, compared to 24\% of male graduates. 

% This study investigates the hypothesis that the presence of male students, often exhibiting higher levels of competitiveness, influences female students' career aspirations. Research shows that women often respond less favorably to competition than men, especially in mixed-gender environments \citep{Niederle2011}.  For example, \citet{Gindi2019} found that while girls demonstrate greater competitiveness in verbal tasks, boys tend to be more competitive in motor and spatial domains. Moreover, \citet{Thomas2014} established a link between competitiveness and career preferences, demonstrating that highly competitive individuals (often male) are more likely to pursue STEM-related disciplines. This suggests that classroom gender composition may affect female students' competitiveness and, consequently, their choice of university majors. A higher proportion of male students could foster a more competitive environment, potentially nudging female students towards STEM fields and deterring them from humanities-related disciplines.

% To test this hypothesis, we analyze a comprehensive dataset encompassing the Integrated School Enrollment System in Colombia \citep{SIMAT}, the National Higher Education Information System \citep{SNIES}, and the Formal Education Survey \citep{EDUC}.  Our empirical strategy employs both fixed-effects models and a staggered difference-in-differences (S-DiD) design. Fixed-effects models allow us to control for unobserved student and school characteristics that might influence both classroom gender composition and major choices. Additionally, the S-DiD approach exploits the natural experiment of schools transitioning from single-sex to co-educational settings, allowing us to estimate the causal impact of changing gender composition on female students' university major choices.

% Our findings reveal that classroom gender composition does indeed influence female students' major choices. We find that a higher proportion of male students in a classroom is associated with a decreased likelihood of female students opting for STEM-related careers.  This suggests that increased male presence might not necessarily encourage female students into STEM fields, as hypothesized. Additionally, we observe varying effects across different academic disciplines, with female students exhibiting a higher preference for traditionally female-dominated fields like social sciences and humanities in classrooms with greater male representation. These findings underscore the complex interplay of gender dynamics, competitiveness, and societal expectations in shaping educational choices.


% This research makes three important contributions. Firstly, it provides the first empirical investigation of the relationship between classroom gender composition and female students' university major choices in Colombia, a developing country with a persistent gender gap in STEM fields \citep{informe2023lee}. While existing research often explores peer gender effects on major choice within specific contexts, such as business schools \citep{zolitz2021effect}, this study examines a broader range of academic disciplines.  Moreover, it moves beyond examining the influence of gender composition on academic performance \citep{EISENKOPF2015123, Pregaldini2020} to explicitly focus on its impact on major selection, providing valuable insights into the mechanisms shaping career aspirations. Secondly, this research examines the nuanced effects of varying gender compositions within mixed-gender classrooms.  This contrasts with studies that examine broader differences between single-sex and co-educational settings \citep{bernal2022effect}, offering a more fine-grained understanding of how classroom gender dynamics affect female students' post-secondary decisions. Finally, this study has important implications for policymakers and educators in Colombia, and potentially other developing countries, seeking to promote gender equality in STEM fields. By understanding the dynamics of gender composition in classrooms, interventions can be tailored to create more inclusive learning environments that foster a more equitable distribution of talent across diverse academic disciplines.

The underrepresentation of women in science, technology, engineering, and mathematics (STEM) fields is a persistent global challenge. Studies consistently highlight the substantial income advantage associated with STEM careers, making these fields highly desirable. For instance, \citet{nsf2023} reports a 35\% income advantage for STEM careers in 2023. However, despite these incentives, women remain less likely than men to pursue STEM education and careers. This disparity limits women's economic opportunities and deprives STEM fields of valuable talent. This paper examines whether the gender composition of classrooms influences female students' choices of university majors, particularly in STEM fields. 

We focus on the case of Colombia, where a significant gender gap persists in STEM fields despite women achieving higher overall university graduation rates. Data from the Instituto Colombiano para la Evaluación de la Educación \citep{ICFES2019} reveals that only 16\% of female university graduates choose STEM careers, compared to 24\% of male graduates. 

This study investigates the hypothesis that the presence of male students, often exhibiting higher levels of competitiveness, influences female students' career aspirations. Research shows that women often respond less favorably to competition than men, especially in mixed-gender environments \citep{Niederle2011}.  For example, \citet{Gindi2019} found that while girls demonstrate greater competitiveness in verbal tasks, boys tend to be more competitive in motor and spatial domains. Moreover, \citet{Thomas2014} established a link between competitiveness and career preferences, demonstrating that highly competitive individuals (often male) are more likely to pursue STEM-related disciplines. This suggests that classroom gender composition may affect female students' competitiveness and, consequently, their choice of university majors. A higher proportion of male students could foster a more competitive environment, potentially nudging female students towards STEM fields and deterring them from humanities-related disciplines.

To test this hypothesis, we analyze a comprehensive dataset encompassing the Integrated School Enrollment System in Colombia \citep{SIMAT}, the National Higher Education Information System \citep{SNIES}, and the Formal Education Survey \citep{EDUC}.  Our empirical strategy employs both fixed-effects models and a staggered difference-in-differences (S-DiD) design. Fixed-effects models allow us to control for unobserved student and school characteristics that might influence both classroom gender composition and major choices. Additionally, the S-DiD approach exploits the natural experiment of schools transitioning from single-sex to co-educational settings, allowing us to estimate the causal impact of changing gender composition on female students' university major choices.

Our findings reveal that classroom gender composition does indeed influence female students' major choices. We find that a higher proportion of male students in a classroom is associated with a decreased likelihood of female students opting for STEM-related careers \textcolor{red}{[value]}.  This suggests that increased male presence might not necessarily encourage female students into STEM fields, as hypothesized. Additionally, we observe varying effects across different academic disciplines, with female students exhibiting a higher preference for traditionally female-dominated fields like social sciences and humanities in classrooms with greater male representation \textcolor{red}{[value]}. These findings underscore the complex interplay of gender dynamics, competitiveness, and societal expectations in shaping educational choices.

This research makes three important contributions. Firstly, it provides the first empirical investigation of the relationship between classroom gender composition and female students' university major choices in Colombia, a developing country with a persistent gender gap in STEM fields \citep{informe2023lee}. While existing research often explores peer gender effects on major choice within specific contexts, such as business schools \citep{zolitz2021effect}, this study examines a broader range of academic disciplines.  Moreover, it moves beyond examining the influence of gender composition on academic performance \citep{EISENKOPF2015123, Pregaldini2020_1} to explicitly focus on its impact on major selection, providing valuable insights into the mechanisms shaping career aspirations. Secondly, this research examines the nuanced effects of varying gender compositions within mixed-gender classrooms.  This contrasts with studies that examine broader differences between single-sex and co-educational settings \citep{bernal2022effect}, offering a more fine-grained understanding of how classroom gender dynamics affect female students' post-secondary decisions. Finally, this study has important implications for policymakers and educators in Colombia, and potentially other developing countries, seeking to promote gender equality in STEM fields. By understanding the dynamics of gender composition in classrooms, interventions can be tailored to create more inclusive learning environments that foster a more equitable distribution of talent across diverse academic disciplines.


The remainder of this article is organized as follows. Section \ref{section:02} presents the data used in this study. The empirical strategy, including the mechanism by which gender composition can affect post-secondary study choices, is presented in Section \ref{section:03}. The results are discussed in Section \ref{section:04}, and the article concludes in Section \ref{section:05}. Robustness checks and additional results are reported in the Appendix. 

% The underrepresentation of women in science, technology, engineering, and mathematics (STEM) fields is a persistent global challenge.  Studies consistently highlight the substantial income advantage associated with careers in STEM compared to other fields. For instance, The \citet{nsf2023} reports a 35\% income advantage for STEM careers, emphasizing the significant financial incentives driving educational choices. Despite this, women remain less likely than men to pursue these fields.  This disparity limits women's economic opportunities and deprives STEM fields of valuable talent. This paper examines whether the gender composition of classrooms influences female students' choices of university majors, particularly in STEM fields. 


% We focus on the case of Colombia, where a significant gender gap in STEM fields persists despite women achieving higher overall university graduation rates. Data from the Instituto Colombiano para la Evaluación de la Educación \citep{ICFES2019} reveals that only 16\% of female university graduates choose STEM careers, compared to 24\% of male graduates.  

% This study explores the hypothesis that the presence of male students, often exhibiting higher levels of competitiveness, exerts a distinct influence on female students' career aspirations and decisions.  Research has shown that women often respond less favorably to competition than men, particularly in mixed-gender environments \citep{Niederle2011}.  \citet{Gindi2019} found that while girls demonstrated greater competitiveness in verbal tasks, boys tended to be more competitive in motor and spatial domains.   Furthermore, \citet{Thomas2014} established a link between competitiveness and career preferences, showing that highly competitive individuals, often male, are more likely to pursue STEM-related disciplines. This suggests that the gender composition of classrooms may influence female students' competitiveness and, consequently, their choice of university majors.  A higher proportion of male students in a classroom could foster a more competitive environment, potentially nudging female students towards STEM fields while deterring them from humanities-related disciplines. 
 
% To test these hypotheses, we analyze a comprehensive dataset encompassing the Integrated School Enrollment System in Colombia \citep{SIMAT}, the National Higher Education Information System \citep{SNIES}, and the Formal Education Survey \citep{EDUC}.  Our empirical strategy leverages both fixed-effects models and a staggered difference-in-differences (S-DiD) design.  The fixed-effects models allow us to control for unobserved student and school characteristics that might influence both classroom gender composition and major choices. We use the S-DiD approach to exploit the natural experiment of schools transitioning from single-sex to co-educational settings, providing a causal estimate of the impact of changing gender composition on female students' university major decisions.



% This research makes three important contributions to the understanding of gender dynamics and educational choices. First, it provides the first empirical investigation of the relationship between classroom gender composition and female students' university major choices in Colombia, a developing country with a persistent gender gap in STEM fields\citep{informe2023lee}. While existing research often explores the effects of peer gender on major choice within specific contexts, such as business schools \citep{zolitz2021effect}, this study takes a broader approach, examining a wider range of academic disciplines.  Additionally, while existing research highlights the influence of gender composition on academic performance \citep{EISENKOPF2015123, Pregaldini2020}, this study explicitly examines its impact on major selection, offering insights into the mechanisms driving career aspirations. Second, this research delves deeper into the impact of varying gender compositions within mixed-gender classrooms. In contrast to studies that focus on the broader differences between single-sex and co-educational settings \citep{bernal2022effect}, this study examines the nuanced effects of different male-to-female ratios within mixed-gender classrooms. This approach provides a more fine-grained understanding of how gender dynamics within the classroom influence female students' post-secondary schooling decisions. Finally, our findings have important implications for policymakers and educators in Colombia and potentially other developing countries seeking to promote gender equality in STEM fields. By understanding the dynamics of gender composition in classrooms, interventions can be tailored to create more inclusive learning environments that foster a more equitable distribution of talent across diverse academic disciplines.


% The remainder of this article is organized as follows. Section \ref{section:02} presents the data used in this study. The empirical strategy, which allows for the estimation of causal inference, including the mechanism by which gender composition can affect post-secondary study choices, is presented in Section \ref{section:03}. The results are discussed in Section \ref{section:04}, and the article concludes in Section \ref{section:05}.  Robustness checks and additional results are reported in the Appendix. 

